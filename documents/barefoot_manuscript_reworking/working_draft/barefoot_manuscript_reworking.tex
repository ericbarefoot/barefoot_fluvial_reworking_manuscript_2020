\documentclass[draft]{compact_proposal}

\usepackage[utf8]{inputenc}
\usepackage[style=authoryear,isbn=false,url=false,maxcitenames=2,uniquelist=false]{biblatex}
\addbibresource{reworking_manuscript.bib}
\usepackage[english]{babel}
\usepackage{fancyhdr}
\usepackage{blindtext}
\usepackage{xfrac}
\usepackage{siunitx}
\usepackage{xspace}
\usepackage[version=4]{mhchem}
\usepackage{wasysym}
\usepackage{gensymb}

\graphicspath{{figures/}}

\newcommand{\del}[3]{\ce{\delta^#1#2_{#3}}}

\title{High channel mobility sequesters coarse sediment in floodplain deposits}
\author{Eric Barefoot}
\date{\today}

\begin{document}

\maketitle % this will do for now, but eventually build a title page from Andrew and Tian's examples.

\section{Abstract}

\section{Introduction}

The PETM altered hydrological cycling

PETM is connected to sedimentological changes on a global scale

  Enrichment in mud on continental margins

  Coarsening of fluvial deposits

  Changes in fluvial sedimentation style

Fluvial systems responded to accommodate altered hydrological cycling, but the mechanisms invoked for this response are ambiguous.
Generally, previous studies invoke increased discharges and transient increases in sediment flux.
The source for this sediment is thought generally to be excess sediment stored in hillslopes.

However, it is not well understood whether increased fluxes of sediment are required to enrich the floodplain in sand.
Rather, channel mobility alone could be the culprit.

Turn to a site which experienced a fluvial response to altered hydrological cycling, and evaluate specific changes in paleohydrology to develop a model for how changes in hydrology can alter sedimentology.

Will address the following hypotheses:

  River channel geometries and kinematics respond to altered hydrology during the PETM via deepening and greater mobility.

  Higher channel mobility drives sand enrichment in floodplain deposits, and depletes the floodplain of mud, which is exported to downstream basins.

\section{Study Site and Methods}


Brief summary of Piceance Basin stratigraphy

It’s a good site for our study because it experienced:

  No documented changes in subsidence history during this interval

  No changes in sediment provenance

  No evidence for tectonically-driven increases in sediment flux

  Paleoshoreline at a great distance (1000s of kilometers)

Paleohydraulic estimates

  Bar clinoform thicknesses

  Grain size

  Reworking index

    Ratio of partially preserved barforms to fully preserved barforms

    Indicates higher relative revisitation rate of channels

Model for sand enrichment via lateral migration*

  Box model approach, where sand fraction in reservoirs is a function of channel kinematics.

  Will likely include feedbacks where enrichment in sand depends on sand fraction.

\section{Results and Interpretations}

No change in paleogeometries
  Partially a function of poor resolution
  Indicate that channel geometries likely did not change by more than a factor of 10
  This implies that rivers did not necessarily carry a higher discharge or sediment flux
More floodplain reworking during PETM interval
  Shown through increased cross-cutting of bar deposits
  Also through changes in avulsion style (hope to add this if we bring in Liz)
  Probably due to flashier hydrograph
  Implies faster lateral migration rates and avulsion rates.
  This process could explain coarsening of deposits without invoking higher fluxes.
Increased sand fraction of fluvial deposits during PETM
  Related to avulsion and lateral migration “combing” or “distilling” of the floodplain, excavating fine material, re-exposing it transport, and distributing sand parcels across the floodplain.
  An individual parcel of floodplain material therefore had a higher likelihood of being reworked, which likely increased the exposure frequency of floodplain organic matter to oxidation
  The converse of this assertion is sand is preferentially bypassed during Atwell and Shire time, likely due to stable channels that act as conduits for coarse sediment.

\section{Conclusions and Implications}


Higher paleodischarges and sediment fluxes have been invoked to explain sedimentological differences in the Piceance and other basin, but we do not find this to be supported by paleohydraulics.

Instead, the same effect can be achieved by varying only the lateral migration rate of the channel, and partitioning more channel sediment to deposits.

Before the PETM, the initial hydrological regime favored meandering channels that seldom avulsed and laterally migrated through bar accretion.

As a result of the PETM, an altered hydrograph increased channel activity, driving more channel sediment to be deposited in floodplains. This could be because of:
    Destabilized banks (less vegetation)
    Increased frequency of avulsion triggers

Therefore, the sedimentological signature of the PETM can be generated with a simple increase in channel mobility.

Implications for carbon cycling

Implications for basin analysis and reservoir connectivity.

Implications for future environmental change

% \begin{figure}[H]
% 	\centering
% 	\includegraphics[width = 0.9\textwidth]{Figure_file.pdf}
% 	\begin{minipage}{0.9\textwidth}
% 	\vspace*{2mm}
% 	\caption{Caption_Here}
% 	\label{fig:short_fig_name}
% 	\end{minipage}
% \end{figure}

\printbibliography

\end{document}

\documentclass[draft]{compact_proposal}

\usepackage[utf8]{inputenc}
\usepackage[style=authoryear,isbn=false,url=false,maxcitenames=2,uniquelist=false]{biblatex}
\addbibresource{reworking_manuscript.bib}
\usepackage[english]{babel}
\usepackage{fancyhdr}
\usepackage{blindtext}
\usepackage{xfrac}
\usepackage{siunitx}
\usepackage{xspace}
\usepackage[version=4]{mhchem}
\usepackage{wasysym}
\usepackage{gensymb}
\usepackage{xcolor}

\graphicspath{{figures/}}

\newcommand{\del}[3]{\ce{\delta^#1#2_{#3}}}

\title{High channel mobility sequesters coarse sediment in floodplain deposits}
\author{Eric Barefoot}
\date{\today}

\begin{document}

\maketitle

% \begin{center}
%   \fcolorbox{gray}{white}{\fcolorbox{gray}{white}{\color{gray} DRAFT MANUSCRIPT FOR SUBMISSION TO GEOLOGY}}
% \end{center}

\section{Abstract}

\section{Introduction}

\pnote[so, so many papers open with "The PETM is the most..." so would prefer to start with a more relevant statement about surface processes controlling stratigraphic architecture of alluvial deposits.]

The Paleocene Eocene Thermal Maximum (PETM) is the most severe climate perturbation known in the Cenozoic Era.
The PETM is associated with altered hydrological cycling and precipitation extremes driven by rapid warming and accumulation of atmospheric carbon.
As such, it serves as the best-known, and best-studied past analogue for contemporary anthropogenic climate warming.
By studying paleo-environmental change during the PETM, we can better understand the mechanisms by which landscapes respond to rapid climate warming.
Insights derived from the paleoenvironmental record of the PETM can be leveraged to guide and improve climate models and long-term hazard forecasts.

The stratigraphic record of the PETM generally indicates that altered hydrological cycling and temperature patterns impacted landscapes and sedimentary systems.
Associated most often with a distinct, abrupt shift in sedimentary facies, the PETM has been identified to coincide with lithostratigraphic contacts globally.
For example, the PETM is connected to increased terrestrial fine-grained flux to marginal marine environments \cnote[miller], coarsening and braiding of fluvial environments \cnote[pujalte], and \pnote[something else..?? or just add BZF to last one.]

Overall, these globally-distributed observations suggest that terrestrial sediment transport systems responded to accommodate altered hydrological cycling, but mechanisms invoked for this response are ambiguous or inconclusive.
Generally, previous studies suggest that increased discharge \cnote[recent Nature Reports], transient increases in sediment flux \cnote[who was I thinking of?], or fluvial steepening \cnote[bzf 2014] are responsible for changes in fluvial style and sediment transport.
However, it has not been demonstrated that changes in total water and sediment flux are necessary conditions for sedimentological changes in fluvial systems during the PETM.

Previous work has generally relied on observations of sedimentary structures that correlate with channel and floodplain geometries, i.e. channel depths, channel widths, bedload grain size, and reach-averaged slope.
Testing hypotheses of fluvial paleohydraulics with geometry-based methods is complicated by uncertainty inherent in the inversion relationships, as well as unknown variance fluvial geometries in the floodplain \pnote[this is vague, so just clarify later that you mean that we don't really know how variable channel depth is across a floodplain in the modern.].
Furthermore, geometric models for paleohydraulics require extrapolation to infer the dynamics of the fluvial system, introducing additional error and uncertainty.
However, recent work has highlighted new methods to directly estimate relative rates of channel processes, or derive a proxy for relative rates of channel kinematics.
When combined with geometric observations, indicators of fluvial paleokinematics may help to capture both the total scale of the system and its behavior as the system evolved through PETM time.

% new tack: previous studies generally use geometry-based paleohydraulic methods or qualitative methods, but these do not tell the whole story, and are typically just partial proxies for fluvial dynamics.
% recent work has highlighted new ways to estimate some aspects of fluvial kinematics, which, when combined with geometric models for paleohydraulics, offers new insight to constrain the fluvial response problem.
% In this study, we combine indicators of geometric and kinematic shifts in the fluvial system in the Piceance Basin as a result of climatic forcing during the PETM.
% Accordingly, it is often difficult in stratigraphic studies to tell the difference between signal and noise.

% it is not well understood whether increased fluxes of sediment are required to enrich the floodplain in sand.
% Rather, channel mobility alone could be the culprit.

Therefore, we return to the Piceance Creek Basin, a field site previously studied as a case study of fluvial response to altered hydrological cycling during the PETM.
Studies of the Piceance Basin to date \pnote[is it just BZF?] have remained equivocal as to the precise mechanisms generating fluvial response at the PETM boundary.
In particular, it is not clear whether fluvial systems experienced net increases in sediment and water flux during the PETM.
Increased fluxes and floodplain diffusivity implies (increasing/decreasing) river gradients and more active channel kinematics, which should be expected to be preserved in the stratigraphic record.
Nonetheless, these predictions have not been tested with paleohydraulic techniques developed in recent years since the initial studies on the Piceance Basin.

In particular, the \cnote[BZF] model predicts that higher sediment and water fluxes generated deeper, coarser channels, and higher floodplain diffusivity with a lower gradient.
As a result, river channel geometries would be expected to grow, and estimates of slope are expected to decrease, while kinematics, or reworking, is expected to increase with a transient increase in sediment flux.
We also consider an alternative case, where mean sediment flux and water discharge is constant, but intra-annual variability in the flow conditions changes during the PETM.

Here, we deploy recently developed paleohydraulic techniques to evaluate hypothesized changes in fluvial geometries and kinematics and develop a new model for fluvial response to climatic perturbation.

\section{Study Site and Methods}

% Brief summary of Piceance Basin stratigraphy

To study the impacts of PETM-modulated changes in hydrological cycling, we turn to the Piceance Creek Basin, an intermontane Laramide depocenter active throughout the Paleogene and Eocene \cnote.
Sedimentary fill of the Piceance Creek Basin is conformable across the Paleocene-Eocene boundary and records climate conditions integrated over approximately \pnote[this much area] km$^2$ of drainage area during the PETM \cnote.
In the Piceance Creek Basin, \del{13}{C}{org} records of the PETM interval correlates with a distinctive sand-rich unit known as the Molina Member of the Wasatch Formation \cnote[bzf].

Juxtaposed between mud-rich units, the Molina Member exhibits starkly different sedimentological characteristics as opposed to the underlying Atwell Gulch Member and the overlying Shire member of the Wasatch Formation.
While the Atwell Gulch and Shire members are composed largely of isolated channel sand bodies confined within a matrix of fine-grained overbank deposits, the Molina member comprises interconnected, sheetlike amalgamated sand bodies with relatively thin intervening layers of floodplain fines.
The distinctive sand enrichment in the Molina member has been attributed to a fluvial response to climate-induced changes to hydrological cycling during the PETM.

However, the precise mechanisms underlying the interpreted fluvial response remain unclear.
The current accepted model invokes a transient increase in water and sediment flux to generate the selective deposition of coarse channel facies in the Molina member \cnote[bzf].
It is difficult to distinguish between this scenario and alternative depositional histories though, because the equivalent signal could feasibly be generated through increased channel dynamics, independent of the net increase in sediment or water supply.
More detailed and precise paleohydraulic methods may shed light on the relative importance of sediment supply and channel dynamics in the Piceance Basin, as these different sources of sedimentological change are predicted to alter the geometrical and kinematic properties of fluvial channels in dinstinctive ways.

The Piceance Creek is a uniquely suited locale for testing hypothesized landscape responses to climate perturbation using fluvial paleohydraulics.
While climate is expected to influence basin dynamics significantly, other key allogenic forcing parameters such as tectonic uplift/subsidence and base level fluctuations have been previously shown to be either negligible or invariant through the PETM interval.
Detrital zircon analysis and formation geometries argue against a sediment supply increase due to tectonic uplift or unroofing of sand-rich units in the hinterland.
Additionally, the great distance of the paleo-shoreline ($>1000$km) and a lack of evidence for temporally-variable subsidence indicate that neither of these allogenic mechanisms are likely to have influenced sedimentary stacking in the Piceance Creek Basin.

% It’s a good site for our study because it experienced:
%
%   No documented changes in subsidence history during this interval
%
%   No changes in sediment provenance
%
%   No evidence for tectonically-driven increases in sediment flux
%
%   Paleoshoreline at a great distance (1000s of kilometers)

A climate-induced change in hydrological cycling, therefore, remains as the most parsimonious explanation for observed facies changes and enrichment of sand in the Molina member.
However, to better understand how fluvial systems responded


Paleohydraulic estimates

  Bar clinoform thicknesses

  Grain size

  Reworking index

    Ratio of partially preserved barforms to fully preserved barforms

    Indicates higher relative revisitation rate of channels

Model for sand enrichment via lateral migration*

  Box model approach, where sand fraction in reservoirs is a function of channel kinematics.

  Will likely include feedbacks where enrichment in sand depends on sand fraction.

\section{Results and Interpretations}

No change in paleogeometries
  Partially a function of poor resolution
  Indicate that channel geometries likely did not change by more than a factor of 10
  This implies that rivers did not necessarily carry a higher discharge or sediment flux
More floodplain reworking during PETM interval
  Shown through increased cross-cutting of bar deposits
  Also through changes in avulsion style (hope to add this if we bring in Liz)
  Probably due to flashier hydrograph
  Implies faster lateral migration rates and avulsion rates.
  This process could explain coarsening of deposits without invoking higher fluxes.
Increased sand fraction of fluvial deposits during PETM
  Related to avulsion and lateral migration “combing” or “distilling” of the floodplain, excavating fine material, re-exposing it transport, and distributing sand parcels across the floodplain.
  An individual parcel of floodplain material therefore had a higher likelihood of being reworked, which likely increased the exposure frequency of floodplain organic matter to oxidation
  The converse of this assertion is sand is preferentially bypassed during Atwell and Shire time, likely due to stable channels that act as conduits for coarse sediment.

\section{Conclusions and Implications}


Higher paleodischarges and sediment fluxes have been invoked to explain sedimentological differences in the Piceance and other basin, but we do not find this to be supported by paleohydraulics.

Instead, the same effect can be achieved by varying only the lateral migration rate of the channel, and partitioning more channel sediment to deposits.

Before the PETM, the initial hydrological regime favored meandering channels that seldom avulsed and laterally migrated through bar accretion.

As a result of the PETM, an altered hydrograph increased channel activity, driving more channel sediment to be deposited in floodplains. This could be because of:
    Destabilized banks (less vegetation)
    Increased frequency of avulsion triggers

Therefore, the sedimentological signature of the PETM can be generated with a simple increase in channel mobility.

Implications for carbon cycling

Implications for basin analysis and reservoir connectivity.

Implications for future environmental change

% \begin{figure}[H]
% 	\centering
% 	\includegraphics[width = 0.9\textwidth]{Figure_file.pdf}
% 	\begin{minipage}{0.9\textwidth}
% 	\vspace*{2mm}
% 	\caption{Caption_Here}
% 	\label{fig:short_fig_name}
% 	\end{minipage}
% \end{figure}

\printbibliography

\end{document}

\documentclass[draft]{compact_proposal}\usepackage[]{graphicx}\usepackage[]{color}
% maxwidth is the original width if it is less than linewidth
% otherwise use linewidth (to make sure the graphics do not exceed the margin)
\makeatletter
\def\maxwidth{ %
  \ifdim\Gin@nat@width>\linewidth
    \linewidth
  \else
    \Gin@nat@width
  \fi
}
\makeatother

\definecolor{fgcolor}{rgb}{0.345, 0.345, 0.345}
\newcommand{\hlnum}[1]{\textcolor[rgb]{0.686,0.059,0.569}{#1}}%
\newcommand{\hlstr}[1]{\textcolor[rgb]{0.192,0.494,0.8}{#1}}%
\newcommand{\hlcom}[1]{\textcolor[rgb]{0.678,0.584,0.686}{\textit{#1}}}%
\newcommand{\hlopt}[1]{\textcolor[rgb]{0,0,0}{#1}}%
\newcommand{\hlstd}[1]{\textcolor[rgb]{0.345,0.345,0.345}{#1}}%
\newcommand{\hlkwa}[1]{\textcolor[rgb]{0.161,0.373,0.58}{\textbf{#1}}}%
\newcommand{\hlkwb}[1]{\textcolor[rgb]{0.69,0.353,0.396}{#1}}%
\newcommand{\hlkwc}[1]{\textcolor[rgb]{0.333,0.667,0.333}{#1}}%
\newcommand{\hlkwd}[1]{\textcolor[rgb]{0.737,0.353,0.396}{\textbf{#1}}}%
\let\hlipl\hlkwb

\usepackage{framed}
\makeatletter
\newenvironment{kframe}{%
 \def\at@end@of@kframe{}%
 \ifinner\ifhmode%
  \def\at@end@of@kframe{\end{minipage}}%
  \begin{minipage}{\columnwidth}%
 \fi\fi%
 \def\FrameCommand##1{\hskip\@totalleftmargin \hskip-\fboxsep
 \colorbox{shadecolor}{##1}\hskip-\fboxsep
     % There is no \\@totalrightmargin, so:
     \hskip-\linewidth \hskip-\@totalleftmargin \hskip\columnwidth}%
 \MakeFramed {\advance\hsize-\width
   \@totalleftmargin\z@ \linewidth\hsize
   \@setminipage}}%
 {\par\unskip\endMakeFramed%
 \at@end@of@kframe}
\makeatother

\definecolor{shadecolor}{rgb}{.97, .97, .97}
\definecolor{messagecolor}{rgb}{0, 0, 0}
\definecolor{warningcolor}{rgb}{1, 0, 1}
\definecolor{errorcolor}{rgb}{1, 0, 0}
\newenvironment{knitrout}{}{} % an empty environment to be redefined in TeX

\usepackage{alltt}

\usepackage[utf8]{inputenc}
\usepackage[style=authoryear,isbn=false,url=false,maxcitenames=2,uniquelist=false]{biblatex}
\addbibresource{reworking_enrichment.bib}
\usepackage[english]{babel}
\usepackage{fancyhdr}
\usepackage{blindtext}
\usepackage{xfrac}
\usepackage{siunitx}
\usepackage{xspace}
\usepackage[version=4]{mhchem}
\usepackage{wasysym}
\usepackage{gensymb}
\usepackage{xcolor}

\graphicspath{{figures/}}

\newcommand{\del}[3]{\ce{\delta^#1#2_{#3}}}

\title{Climate-driven increases in channel mobility sequesters \\ coarse sediment in terrestrial floodplain deposits}
\author{Eric Barefoot, Jeff Nittrouer, Brady Foreman, Gerald Dickens, \textit{Liz Hajek}}
\date{\today}



\IfFileExists{upquote.sty}{\usepackage{upquote}}{}
\begin{document}

\maketitle




% \begin{center}
%   \fcolorbox{gray}{white}{\fcolorbox{gray}{white}{\color{gray} DRAFT MANUSCRIPT FOR SUBMISSION TO GEOLOGY}}
% \end{center}

\section{Abstract}

\section{Introduction}

Climate is a key control on the flux of solutes and sediment from terrestrial environments to basins on continental margins, the primary archives holding records of paleoenvironmental change.
Stochastic sediment transport processes distort and modify the signal of changing climate conditions as it propagates through terrestrial environments to marine basins, precluding a straightforward interpretation of the stratigraphic record to infer past climate.
A detailed accounting of the mechanisms by which climate alters sediment production, transport, and deposition in terrestrial environments can therefore inform models of marine basin filling during periods of rapid climate change.

Alluvial basins form the central link in the terrestrial sediment transport system, with climate modulating the supply of sediment and water from upstream sources, thereby modifying river dynamics like channel geometries and avulsion rates. 
The volume and character of sediment exported from alluvial basins to downstream systems is controlled by the balance between subsidence and fluvial dynamics.
When rivers are large, and migrate or avulse rapidly relative to subsidence rates, the channel can rework the floodplain surface before sediment subsides below a characteristic compensation depth.
Floodplain stratigraphy built by active rivers that re-excavate floodplain deposits are characterized by densely stacked channel deposits.

It follows that since channel deposits are primarily composed of sand, deep channelized systems with accelerated kinematic rates preferentially fractionate sand into alluvial strata, in the extreme case depriving shorelines of sand and bypassing mud to marine basins.
However, inferring a change in the fractionation of mud and sand in the stratigraphic record and interpreting it as a climatic signal is challenging, as a broad range of terrestrial processes could promote sand enrichment in alluvial floodplains.
For example, an increase in the fraction of sand supplied by hillslopes or a decrease in subsidence rate are both sufficient to enrich sand in the floodplain without requiring a change in fluvial dynamics or climate.

To isolate the influence of climate on sand fractionation in alluvial systems, it is necessary to place independent constraints on subsidence and sediment provenance as well as channel dimensions and kinematics.
Here we apply this approach to the well-studied Piceance Basin (Colorado, USA), a Laramide intermontane basin where changes in channel stacking patterns and sand enrichment have been attributed to altered hydrological cycling related to the Paleocene-Eocene Thermal Maximum (PETM).
Subsidence and sediment provenance are well understood for the Piceance Basin, so we constrain channel geometries and kinematics with a new suite of paleohydraulic data to offer a refined model for alluvial response to abrupt climate warming and explore the impacts of sediment fractionation in alluvial systems on sediment export during the PETM.

\section{The PETM in the Piceance Basin}

The Paleocene Eocene Thermal Maximum (PETM) is the most severe climate perturbation known in the Cenozoic Era \parencite{zachos_early_2008, slotnick_large_2012}.
Associated with altered hydrological cycling and precipitation extremes driven by rapid warming and accumulation of atmospheric carbon \parencite{carmichael_hydrological_2017}, the PETM serves as the best-studied analogue for contemporary climate warming.
Insights derived from the paleoenvironmental record of the PETM can be leveraged to guide and improve climate models and long-term hazard forecasts.

Active throughout the Paleogene and Eocene, the Piceance Basin is an intermontane Laramide basin.
Alluvial deposits there are conformable across the Paleocene-Eocene boundary and record climate conditions integrated over approximately 60,000 km$^2$ of drainage area during the PETM \cnote[johnson].
The PETM interval is identified in the Piceance Basin by a distinctive \del{13}{C}{} isotope excursion, which correlates with a distinctive sand-rich unit known as the Molina member of the Wasatch Formation \parencite{foreman_fluvial_2012}.

Juxtaposed between mud-rich units, sedimentological characteristics of the sandy Molina member contrast starkly with the underlying Atwell Gulch Member and the overlying Shire member of the Wasatch Formation.
While the Atwell Gulch and Shire members are composed largely of a matrix of muddy floodplain paleosols containing isolated channel sand bodies, the Molina member comprises interconnected, sheetlike amalgamated sand bodies with relatively thin intervening layers of floodplain mud.
The distinctive sand enrichment in the Molina member has been explained as a fluvial response to climate-induced changes in hydrological cycling during the PETM \parencite{foreman_fluvial_2012}.

This interpretation connects the response in the Piceance with that in other basins around the world, where sedimentological shifts have been interpreted to be due to more intense monsoonal precipitation during the PETM and enhanced sediment flux from hillslopes \cnote[pujalte, schmitz\&pujalte, foreman, foreman2014].
A sustained increase in sediment flux to alluvial basins can drive disparate responses in alluvial systems, where river channel geometries like width, depth, and slope would have adjusted to accommodate higher sediment fluxes and more intense flows.
The nature of the response carries implications for sediment dispersal throughout the terrestrial transport system.
For example, systems that developed higher gradients would accumulate coarse facies proximal to mountain uplifts, whereas a shallowing in gradient would allow coarse facies to prograde distally.
Evaluating the effects of PETM precipitation on sediment fluxes and distinguishing between different scenarios of sediment dispersal requires constraints on key paleohydraulic parameters.
However, while \textcite{foreman_fluvial_2012} indicate that Piceance Basin rivers deepened and widened to convey higher flows, data from other basins are more equivocal \parencite{chen_estimating_2018}. 
It is even less clear how river gradients or avulsion rates responded to the shift in flooding regime, key parameters for reconstructing paleolandscape dynamics.

The Piceance Creek Basin is a uniquely suited locale in which to test hypothesized histories of landscape responses to climate perturbation using paleohydraulic techniques. 
While climate varies throughout the Wasatch Formation, detrital zircon analysis and basin-scale deposit geometries argue against a wholesale increase in sediment supply due to tectonic uplift or unroofing of sand-rich units in the hinterland.
Additionally, the great distance of the paleo-shoreline ($>1000$km) and a lack of evidence for temporally-variable subsidence indicate that neither of these allogenic mechanisms are likely to have influenced sedimentation in the Piceance Basin.

We use recently developed paleohydraulic techniques to quantify three key paleohydraulic parameters in the Piceance Creek Basin.
Using field data collected during field campaigns from 2017-2019, we quantify changes in flow depth through the Wasatch Formation by measuring the relief on fully-preserved fluvial barforms. 
These paleo-flow depths were combined with co-located measurements of median bedload grain size interpreted using a hand lens and grain-size card to predict the paleoslope via the empirical scaling in \textcite{trampush_empirical_2014}.
We also collected drone imagery for 12 outcrops in the Piceance Creek Basin, and mapped barform preservation and sand preservation throughout the Wasatch Formation. 
These data were used to estimate a relative increase in reworking due to changes in channel kinematics, after \textcite{chamberlin_using_2019} 
Finally, we estimate relative sand bypass and fractionation across the PETM in the Piceance Creek Basin.
\mnote[The dataset of paleoflow depths was augmented with additional data from previous literature and unpublished field work in the Piceance Basin, as well as data documenting avulsion styles in the Wasatch Formation.]{This is what we would like Liz to contribute.}

% 
% sedimentological observations do not  hydraulics are needed to confirm increased 
% 
% However, the expression of the fluvial responses in the geometries and kinematics of Piceance rivers the interpreted fluvial response remain unclear.
% 
% analysis of paleosol geochemistry indicates that in the Piceance Basin, mean annual precipitation actually decreased, but extreme discharge events became more common and intense \parencite{lesko_increased_2019}.
% 
% 
% This scenario implies a concomitant steepening of alluvial channels, as well as an 
%  change model invokes a transient decrease in mean annual water and sediment flux with \pnote[deeper channels and higher intensity sediment transport] to generate the selective deposition of coarse channel facies in the Molina member \cnote[bzf].
% It is difficult to distinguish between this scenario and alternative depositional histories though, because the equivalent signal could feasibly be generated without the net increase in sediment or water supply.
% More detailed and precise paleohydraulic methods may shed light on the relative importance of sediment supply and channel dynamics in the Piceance Basin, as these different sources of sedimentological change are predicted to alter the geometrical and kinematic properties of fluvial channels in distinctive ways.
% 
% \newpage
% 
% 
% 
% \section{Paleohydraulics of alluvial response}
% 
% Changes in runoff and discharge are expected to affect the flow depth and slope of rivers by altering the sediment and water supply.
% Study of paleosol deposits has indicated that in the Piceance Basin, mean annual precipitation may have decreased, but upper-stage sedimentary structures \cnote and floodplain chemistry \cnote suggests that the intra-annual variability of runoff increased \cnote.
% In response, previous work shows that alluvial systems grew deeper and wider during the PETM interval.
% No estimate of paleoslope has been made though, making an interpretation of changing channel geometries and sedimentary structures challenging.
% 
% Altered hydrological conditions are also expected to change the rate of river kinematics like avulsion and lateral migration.
% The Molina Member, deposited during the PETM, is often described as ``amalgamated'' or ``reworked,'' \cnote suggesting that rivers were more active and revisited locations across the floodplain more often.
% This observation holds with a hypothesized increase in discharge variability and lower mean annual precipitation, but is yet to be tested in the field.
% 
% To test the impact of changing climate during the PETM on the alluvial system in the Piceance Basin, we employ a toolkit of paleohydraulic techniques, including two methods to reconstruct bankfull channel depths, one method to reconstruct paleoslope, and a new technique for estimating reworking by avulsions and lateral migration.
% The main considerations for choosing paleohydraulic inversion techniques included a robust accounting of uncertainty in the model framework, and simple application to ancient deposits.
% When possible, estimates of sedimentary geometries were obtained by field measurement and grain size was estimated from hand samples.
% On high cliff faces, and in difficult-to access areas, three-dimensional computer models of outcrops were derived from drone imagery and used to map bar clinoform structures, channel-fill structures, and bedding surfaces.
% 
% We estimate paleo-flow depths by measuring the total relief of fully preserved bar clinoforms \cnote[what is the old citation on this?].
% When outcrops lacked fully preserved bar strata, we estimate flow depth by measuring the distribution of cross-stratification thicknesses, and using empirical relations to convert dune cross-stratification to flow depth, with associated uncertainty.
% Flow depth estimates, and grain size information for interpreted bedload deposits are combined in an empirical relation from \cnote[trampush] to derive an estimate of paleoslope.
% Finally, the relative proportions of fully-preserved, partially preserved, and truncated bar strata were estimated to derive an index of fluvial reworking \cnote.

\section{Results and Interpretations}

% No change in paleogeometries
%   Partially a function of poor resolution
%   Indicate that channel geometries likely did not change by more than a factor of 10
%   This implies that rivers did not necessarily carry a higher discharge or sediment flux

Despite clear differences in sedimentology between the Molina Member and the bounding Atwell Gulch and Shire members, our measurements from the Piceance Creek indicate that paleoflow depths were \mnote[statistically stationary]{Something like this\ldots} across the PETM boundary (see Figure \ref{fig:paleo_geom}a).
Paleoslope estimates also indicate no statistical difference between the river gradients in the Molina as compared to the Atwell Gulch or Shire members (see Figure \ref{fig:paleo_geom}b). 
Taken together, these results suggest that within the resolution of current paleohydraulic methods, rivers in the Piceance Basin did not necessarily carry an increased sediment flux or higher discharge as a result of changes to regional climate and hydrological cycling.

% Paleohydraulic estimates are, however, difficult to constrain in outcrop with high confidence, as the techniques rely on empirical relations developed from modern systems with associated uncertainty.
% Statistical inference for paleoslope, for example, is limited in resolution to a factor of 10 \cnote.
% It is therefore feasible that paleoslope in the Piceance Basin could have changed by less than a factor of ten, and the current state of the art methods would not be able to detect the change.
% The poor resolution of techniques based on channel geometries alone limits inference in the ancient record, and prompts the refinement of existing paleohydraulic models and the development of new ones.
% 
% It is important to note that the poor resolution of paleohydraulic methods only preclude interpretation that the paleoslope in the Piceance Basin did not increase by more than a factor of ten.
% Previous work suggested that increased sediment flux due to hillslope clearing and vegetation overturn could changed slope in the Piceance Basin.
% This remains a possibility, however, as a substantial increase of even a factor of two or five in fluvial slope to accommodate increased sediment supply would go undetected using current methods.
% Similarly, even amongst rivers with comparable depth, \mnote{do I have a dataset to back this up? Will Sheila's provide this info?} discharge can vary by up to a factor of 5, suggesting that inferences based on paleo-depths in the stratigraphic record are likely insensitive to even doubling of discharge.
% A change in channel kinematics is therefore the most likely primary agent for changing sediment partitioning in the Piceance Basin because there is affirmative evidence for this process.
% However, it is feasible that adjustments in total discharge and sediment flux occurred in the Piceance Basin, but they cannot presently be detected with confidence.

% More floodplain reworking during PETM interval
%   Shown through increased cross-cutting of bar deposits
%   Also through changes in avulsion style (hope to add this if we bring in Liz)
%   Probably due to flashier hydrograph
%   Implies faster lateral migration rates and avulsion rates.
%   This process could explain coarsening of deposits without invoking higher fluxes.

% With no strong evidence for increased total fluxes of water or sediment in channel geometries, we turn to the complimentary question of interpreting channel paleokinematics.
% We employ a newly developed technique that quantifies the interrelationship and reworking of fluvial structures, thereby \mnote[attacking]{better word} the kinematics of fluvial channels during the PETM in the Piceance Basin.
% Drone imagery was collected from \pnote[how many?] outcrops throughout the Shire, Molina, and Atwell Gulch members of the Wasatch Formation and was processed using photogrammetry to generate 3D models of outcrops.
% This drone imagery was interpreted using \pnote[may use the software.] and barform structures identified and mapped. geometric information was collected for all barforms, as well as their interrelationships and degree of preservation.
% \mnote{Should all of this go in the methods?}

Mapping sedimentary structure preservation throughout the Wasatch Formation shows that barforms in the Molina member crosscut each other more frequently, and are less well preserved than in either the Shire or Atwell Gulch members.
Our data also indicate that avulsion style changes during Molina time, when \mnote[progradational/incisional]{I think incisional is the preferred based on my observations} avulsions were more abundant. 
Given that barform geometries indicate no change in hydraulic geometry, more abundant cross-cutting and incisional avulsions imply that river channels revisited locations frequently, cross-cutting previously deposited structures and reworking the floodplain surface.

We take our estimates of bar preservation in the Wasatch Formation, and compare them to a suite of numerical experiments conducted using the model developed in \textcite{chamberlin_interpreting_2015}. 
This reduced-complexity model allows avulsion style, as well as key parameters like avulsion rate, channel dimensions, and subsidence to be controlled independently to produce synthetic stratigraphy. 
In the suite of experiments, we varied avulsion rate while holding channel dimensions and other boundary conditions constant, and measured the preservation of channel elements as a proxy for bar preservation. 
This approach is adapted from \textcite{chamberlin_using_2019}, and we quantify the amount of floodplain reworking as well as the concomitant sand enrichment.

As channel deposits are generally composed of sand, more rapid channel kinematics enriches floodplains in sand, and preferentially re\"entrains mud.
In our model runs, when avulsion rate increases, channel deposits (sand) occupy more of the stratigraphy, and cross-cut other channel deposits more frequently. 
To compare our model results with observations in the Piceance Basin, we construct a statistical model from our numerical results, allowing us to use bar preservation measured in outcrops to predict both total sediment preservation in a given cross-section, as well as sand enrichment.

Applying our new statistical model to estimates of bar preservation in the Wasatch predicts in the Atwell Gulch and Shire members, X\% of the original sediment is preserved and Y\% of the deposit would be composed of sand given an input sand concentration of Z\%. 
In contrast, assuming the same input concentration of sand, our model predicts that P\% of sediment is preserved in the Molina member, leaving a total concentration of Q\% sand. 
The transition from Y\% sand in the Atwell Gulch and Shire members to Q\% in the Molina represents a (Q-Y)\% increase in concentration due to a (X-P)\% increase in reworking.

Basin-wide estimates of sediment thickness indicate that whereas before and after the PETM, floodplain deposits were composed of approximately \mnote[20\%]{check} coarse material, during the PETM far more sand (up to \mnote[40\%]{check}) makes up floodplain deposits.
This overall doubling in sand concentration agrees well with the estimates from our statistical model, and suggest that reworking by accelerated avulsion rates and floodplain reworking may be responsible for the increased floodplain sand fraction of the Molina member rather than an increase in total sediment supply or discharge.
This inference is supported by a lack of observable change in channel geometries indicative of changing flux conditions, like adjustments to channel depth or slope.
Therefore we offer here a new interpretation of landscape response to shifting hydrolgical regime, whereby increased seasonality of discharge during the PETM weakened channel banks, allowing more rapid channel migration and altered in-channel sedimentation shortened avulsion timescales.

% 
% 
% supporting other evidence of accelerated channel kinematics during the PETM interval.
% 
% We measure the ratio of fully-preserved barforms to truncated barforms in the Wasatch Formation, and 
% 
% These results too suggest more rapid channel kinematics, and more frequent incisional avulsions relative to subsidence rates.
% As a result, the average residence time of sediment in floodplain deposits would have decreased, exposing fine sediments to more frequent transport in alluvial channels.
% In this way, channels excavated previously deposited overbank material, which preferentially deposited channel bodies in the floodplain.





*these prob belongs in the next section.
  An individual parcel of floodplain material therefore had a higher likelihood of being reworked, which likely increased the exposure frequency of floodplain organic matter to oxidation
  The converse of this assertion is sand is preferentially bypassed during Atwell and Shire time, likely due to stable channels that act as conduits for coarse sediment.

\section{Conclusions and Implications}

Changes in paleodischarge and sediment flux has been previously invoked to explain sedimentological differences in the Piceance Basin, as well as other localities.
However, in the Piceance Basin, we do not find this interpretation to be strictly supported by paleohydraulic estimates of fluvial channel geometries.
Our results suggest that sedimentological shifts during the PETM are instead driven by changes in channel kinematic rates.
We propose that increased seasonality of precipitation during the PETM is responsible for sequestering more channel sediment in floodplain deposits by varying the lateral migration rate and avulsion timescales of the channel system.

Before the PETM, the \mnote[initial]{not a great word choice} hydrological regime favored meandering channels that seldom avulsed and laterally migrated through bar accretion.
The outcome of this style of fluvial system is the preferential deposition of channel-derived sand bodies in thin ribbons.
Low precipitation seasonality corresponds to low flooding intensity in the river system, a condition thought to induce strong, confining river banks that promote low width-to-depth ratio channels that migrate slowly an avulse seldomly.
A low width-to-depth ratio induces high shear stresses on the bed, ensuring that coarse sediment is transported primarily in the thalweg, and low Froude numbers promote \pnote[something].
Accordingly, with high bank strength, due both to vegetation density and cohesive material, channels would have migrated laterally at reduced rates, and increased shear stress year-round on the bed would have reduced in-channel sedimentation, forestalling superelevation of the riverbed, and therefore forcing less frequent avulsions.

In contrast, as a result of the PETM, an more intense flooding regime and peakier hydrograph increased channel activity, driving more channel sediment to be deposited in floodplains.
The primary mechanisms for increasing channel kinematic rates is through destabilized banks.
It is thought that decreased riparian vegetation density via seasonal drying and more intense flood peaks worked in tandem to increase the width-to-depth ratio of PETM rivers in the Piceance Basin.
Higher width-to-depth ratios in PETM rivers probably raised local Froude numbers of flows, potentially into critical, or even super-critical flows during flood stages.
This increases the potential for significant in-channel sedimentation during low flows, and less confining ability for the channel during high flows.
Both the infilling of channels during low flow, and low confinement during high flow, increase the likelihood of avulsion, thereby increasing avulsion frequency.

Our integrated observations of paleohydraulics favor a scenario where Piceance river systems were slow-moving and confined by strong banks during Atwell Gulch and Shire time, before and after the PETM.
In contrast, during the PETM Piceance river systems had weak banks and moved rapidly across the floodplain surface during Molina time.
The result of this see-saw change in fluvial system dynamics across the PETM was a change in the relative partitioning of coarse and fine sediment into floodplain deposits.

When river banks were strong, i.e. in Atwell Gulch and Shire times, low intensity flooding in Piceance rivers allowed only fine-grained silt and mud to be transported over levees to the floodplain.
Coarser sand was instead transported in the thalweg, and low lateral migration rates relative to subsidence rates meant that thalweg deposits constitute a smaller fraction of the overall deposit.
The result was that during Atwell Gulch and Shire time, river channels were most likely long-lived stable conduits conveying coarse material through the system without depositing it in the floodplain.
In this way, we can consider Atwell Gulch and Shire rivers to be efficient transporters of sand, preferentially retaining mud and silt in the floodplain while allowing sand to bypass the depocenter.

Correspondingly, during Molina time, weak river banks and rapid channel kinematics relative to subsidence rates ensured that channels visited locations across the floodplain surface frequently, and that flood-stage flows could transport coarse sediment more effectively to the overbank environment.
More frequent visitation of the river across the floodplain surface likely meant that deposits were re-excavated more frequently than during Atwell Gulch or Shire time, and fine silt and mud deposits were more likely to be re-mobilized during Molina time and replaced by coarse sand.
Additionally, a higher incidence of crevasse splays and overbank coarse deposition preferentially sequestered coarse material in non-channel deposits.
The integrated effect of these accelerated processes was that coarse sandy sediment was retained more often in the floodplain.
As such, Piceance rivers during Molina time were inefficient transporters of sand, preferentially bypassing fine-grained silt and mud to downstream basins.

These results imply that hydrologically-modulated channel dynamics affect the distribution of sediment grain-size fractions retained in continental depocenters in unintuitive ways.
Accelerated rates of lateral migration, for instance, have traditionally been thought to increase the diffusivity of floodplains, allowing coarser facies to prograde further into the basin system.
Accordingly, lower rates of channel kinematics would be expected to inhibit the transport of coarse facies to distal portions of the basin.
In fact, our finding suggest that the opposite may be true, and during periods of dampened channel kinematics, coarse sediment is actually preferentially bypassed through the system, and deposited even more distally than with active channels.

The preferential partitioning of grain-size fractions due to channel kinematic rates could have implications for understanding carbon cycling.
As recent studies have highlighted \cnote[Lyons], the PETM is associated with a sustained period of \del{13}{C}{}-depleted carbon release, which they hypothesize could be due to increased export of fossil carbon from soils.
The primary evidence for this claim is the increased accumulation of fossil carbon and increased accumulation of fine-grained sediment in marine sequences.
However, \cnote[Lyons] do not invoke a precise mechanism for the increased reworking of soil fossil carbon, only stating that by some process, floodplain material is reworked to expose fossil soil carbon to respiration.
Our observations suggest that increased seasonality of precipitation is a viable mechanism to promote increased reworking of fossil carbon reservoirs.
In the process of excavating fine-grained silt and sand reservoirs, Molina rivers would have also re\"exposed carbon stored in those floodplain deposits to respiration.
We propose that a global increase in precipitation seasonality may be the mechanism driving the excess flux in soil carbon after the initial PETM onset.

In certain circumstances, the impact of climate modulation on channel dynamics may adjust conventional notions of where petroleum reservoirs are emplaced along the source-to-sink continuum.
For example, a system under a highly seasonal hydrology would be expected to partition sand preferentially in the proximal parts of the source-to-sink system, and off-shore deposits would be expected to be rich in mud.
As a result, knowledge of climate conditions during exploration could indicate that most reservoir-forming deposits may be located in terrestrial reservoirs during that interval.
Additionally, reservoirs emplaced by laterally mobile systems like Molina river systems will exhibit greater reservoir connectivity due to the frequent interaction of sand bodies in the formation.

The absolute timing of fluvial response as it relates to the onset of the PETM is difficult to constrain in the stratigraphic record.
However, it is clear that channel dynamics in the Piceance Basin adjusted rapidly to new hydrological conditions.
As humanity prepares for environmental impacts due to anthropogenic climate change, it will be important to consider the medium and long-term consequences for infrastructure situated on floodplains.
Higher-intensity flooding in some regions may drive significant increases in channel mobility, threatening adjacent floodplain communities.
Increased risk of avulsions due to changes in hydrological cycling would also present a threat to communities.

\pnote[how to wrap up? Come up with some kind of good ending idea.]

% %====================%%
% % Maybe move everything in here down to the discussion when you talk about global sediment partitioning.
% %====================%%
% By studying paleo-environmental change during the PETM, we can better understand the mechanisms by which landscapes respond to rapid climate warming.
% %====================%%
% The stratigraphic record of the PETM generally indicates that altered hydrological cycling and temperature patterns impacted landscapes and sedimentary systems.
% Associated most often with a distinct, abrupt shift in sedimentary facies, the PETM has been identified to coincide with lithostratigraphic contacts globally.
% For example, the PETM is connected to increased terrestrial fine-grained flux to marginal marine environments \cnote[miller], coarsening and braiding of fluvial environments \cnote[pujalte], and \pnote[something else..?? or just add BZF to last one.]
% %====================%%
% Overall, these globally-distributed observations suggest that terrestrial sediment transport systems responded to accommodate altered hydrological cycling, but mechanisms invoked for this response are ambiguous or inconclusive.
% Generally, previous studies suggest that increased discharge \cnote[recent Nature Reports], transient increases in sediment flux \cnote[who was I thinking of?], or fluvial steepening \cnote[bzf 2014] are responsible for changes in fluvial style and sediment transport.
% However, it has not been demonstrated that changes in total water and sediment flux are necessary conditions for sedimentological changes in fluvial systems during the PETM.
% %====================%%


Previous work has generally relied on observations of sedimentary structures that correlate with channel and floodplain geometries, i.e. channel depths, channel widths, bedload grain size, and reach-averaged slope.
Testing hypotheses of fluvial paleohydraulics with geometry-based methods is complicated by uncertainty inherent in the inversion relationships, as well as unknown variance fluvial geometries in the floodplain \pnote[this is vague, so just clarify later that you mean that we don't really know how variable channel depth is across a floodplain in the modern.].
Furthermore, geometric models for paleohydraulics require extrapolation to infer the dynamics of the fluvial system, introducing additional error and uncertainty.

However, recent work has highlighted new methods to directly estimate relative rates of channel processes, or derive a proxy for relative rates of channel kinematics.
When combined with geometric observations, indicators of fluvial paleokinematics may help to capture both the total scale of the system and its behavior as the system evolved through PETM time.

% new tack: previous studies generally use geometry-based paleohydraulic methods or qualitative methods, but these do not tell the whole story, and are typically just partial proxies for fluvial dynamics.
% recent work has highlighted new ways to estimate some aspects of fluvial kinematics, which, when combined with geometric models for paleohydraulics, offers new insight to constrain the fluvial response problem.
% In this study, we combine indicators of geometric and kinematic shifts in the fluvial system in the Piceance Basin as a result of climatic forcing during the PETM.
% Accordingly, it is often difficult in stratigraphic studies to tell the difference between signal and noise.

% it is not well understood whether increased fluxes of sediment are required to enrich the floodplain in sand.
% Rather, channel mobility alone could be the culprit.

Therefore, we return to the Piceance Basin, a field site previously studied as a case study of fluvial response to altered hydrological cycling during the PETM.
Studies of the Piceance Basin to date \pnote[is it just BZF?] have remained equivocal as to the precise mechanisms generating fluvial response at the PETM boundary.
In particular, it is not clear whether fluvial systems experienced net changes in sediment and water flux during the PETM.
Increased fluxes and floodplain diffusivity implies (increasing/decreasing) river gradients and more active channel kinematics, which should be expected to be preserved in the stratigraphic record.
Nonetheless, these predictions have not been tested with paleohydraulic techniques developed in recent years since the initial studies on the Piceance Basin.

In particular, the \cnote[BZF] model predicts that higher sediment and water fluxes generated deeper, coarser channels, and higher floodplain diffusivity with a lower gradient.
As a result, river channel geometries would be expected to grow, and estimates of slope are expected to decrease, while kinematics, or reworking, is expected to increase with a transient increase in sediment flux.
We also consider an alternative case, where mean sediment flux and water discharge is constant, but intra-annual variability in the flow conditions changes during the PETM.

Here, we deploy recently developed paleohydraulic techniques to evaluate hypothesized changes in fluvial geometries and kinematics and develop a new model for fluvial response to climatic perturbation.

% To study the impacts of PETM-modulated changes in hydrological cycling, we turn to the Piceance Basin,  \cnote.

A climate-induced change in hydrological cycling, therefore, is the most parsimonious explanation for observed facies changes and enrichment of sand in the Molina member.
However, to better understand how the mechanisms by which fluvial systems responded to altered hydrological cycling, it is necessary to distinguish between a scenario involving increased sediment and water flux and a scenario where average sediment supply and discharge were constant, but channel dynamics were different.
We will distinguish between these scenarios by employing detailed paleohydraulic reconstructions of Molina fluvial systems as contrasted with Atwell Gulch and Shire alluvial systems.
The main geometric attribute that is hypothesized to distinguish between these two scenarios are the size of channel deposits (particularly depth) and the reach-averaged slope of channels.



\mnote[We connect our estimates of paleohydraulic parameters to a simple box-model approach to determine how sand enrichment in the floodplain should vary as a function of channel dynamics versus sediment supply.]{are we actually going to use this?}
This model takes as parameters fluvial sediment flux, grain size distribution (sand v mud), basin geometries, and subsidence rate.
Floodplain sediment composition is determined in this model by the relative proportion of channel deposits vs overbank sediments, each of which have a predetermined sand concentration, where channels have a higher enrichment in sand.
Channel sand-bodies rework overbank sediments, replacing them with channel deposits.
By varying channel geometries and kinematic rates relative to subsidence rates, channel deposits occupy more or less of the morphodynamic reworking zone.
In this way, based on observations of the total enrichment in sand, the abundance of channel deposits, and the frequency of cross-cutting relationships, inferences are made of channel geometries and kinematic rates as a function of floodplain parameters.



% \begin{figure}[H]
% 	\centering
% 	\includegraphics[width = 0.9\textwidth]{Figure_file.pdf}
% 	\begin{minipage}{0.9\textwidth}
% 	\vspace*{2mm}
% 	\caption{Caption_Here}
% 	\label{fig:short_fig_name}
% 	\end{minipage}
% \end{figure}

\printbibliography

\end{document}


\newpage

 subjected to higher sediment and water flux are predicted to experience enhanced progradation of coarse facies as compared a system with lower fluxes.
Assuming constant subsidence rates, stratigraphy rich in coarse-grained channel deposits and depleted in fine-grained overbank floodplain deposition would be interpreted as representing a system with increased sediment and water supply. 

However, other mechanisms exist to enrich floodplain stratigraphy in coarse-grained sediment.
For example, river systems 


 transferring sediment from hillslopes to marine basins, while the alluvial basin itself fills by subsidence and aggradation, recording climatically-induced shifts in landscape dynamics.
 
Rivers with higher fluxes of sediment and water may 

rivers with highly variable fluxes

 to 

deeper, migrate laterally more rapidly, and avulse more frequently

The stratigraphic ar of alluvial basins reflects a balance between the generation of accommodation space and

 geometry and kinematics of alluvial channels the sedimentological 


Climate-modulated changes precipitation patterns have been implicated in changes to channel dynamics in alluvial basins \cnote.
However, the physical mechanisms by which alluvial basin systems respond to altered precipitation


Terrestrial alluvial basins buffer and modify climatic signals 


In particular, changes in climate may alter the processes by which sediment is buffered in alluvial floodplains, and therefore change the composition and supply of sediment exported to continental margins.



landscape evolution and sediment transport 

An  depends on 

The grain-size composition of terrestrial sediments supplied to continental margins is a first-order control on the grain-size composition of marine stratigraphy.
Terrestrial sediment grain-size composition is controlled by: (1) the distribution of sediment supplied by erosional hillslope environments, (2) comminution of sediment particles through transport, and (3) sorting of grain sizes by sediment transport systems. 
Records of paleoclimate stored in continental margin sediments are therefore impacted by the integrated response of these processes to changes in climate.

On short time-scales ($<1$My), climate has limited influence on the distribution of grain sizes supplied by hillslopes \cnote and likely does not influence the comminution of particles through transport except in extreme cases (e.g. a transition from fluvial to glacial transport) \cnote.


The composition and architecture of sedimentary basins is a product of 

Alluvial sedimentary basins are the primary archives that hold terrestrial records of paleoenvironmental change.
The dominant controls on the composition and character of alluvial strata are the interactions between accommodation space, sediment supply, and channel dynamics.
While the balance between subsidence and sediment supply is generally controlled on long timescales by tectonic factors, climate can impact the 

, most notably by controlling the concentration and distribution of sand and mud in the strata.
Environmental 

reflected in geochemical and biostratigraphic proxies are translated into stratigraphy through \mnote[net]{necessary or not?} deposition.
The fidelity and character of the record is affected, however, by how severely the signal is filtered by sedimentary processes, which can in turn be affected by the environmental change they are recording.
Untangling this linkage between environmental forcings and sedimentary processes is a critical step for interpreting the stratigraphic record of past environmental change.

Climate-modulated changes in water and sediment supply have been shown to be responsible for adjustments in fluvial channel kinematic rates; in particular the rate of channel migration and avulsion \cnote[bryant and others].
The rate of channel kinematics relative to the rate of overall subsidence controls the degree to which channel facies are distributed within the floodplain, and thus the rock record.
More active rivers which migrate quickly and avulse frequently relative to subsidence will accordingly develop stratigraphy rich in channel deposits and coarse material.

While the balance of subsidence and channel kinematics is one control on the stratigraphic arrangement of channel facies, there are many different candidate sedimentary and tectonic processes that can alter this balance.
The particle size distribution of sediment supplied to the alluvial system, for example, can promote changes in channel morphology.
If the ratio of sand to mud is increased, the width-to-depth ratio and migration rate of channels will increase, leading to more fine floodplain material being replaced by coarse deposits.
Channel depth can also change the degree of reworking, where deep river systems excavate and replace fine material, especially if the net subsidence rate is slow relative to channel kinematic timescales.
Every one of these processes can lead to changes in the density of fluvial channels, the degree of stratigraphic reworking, and the fidelity of geochemical records.

Accurately reconstructing environmental signals in alluvial basins therefore requires identifying the combination of processes that affect signal preservation from amongst the wide range of candidates.
In theory, this means that each process which could affect signal preservation must have independent constraints.
However, it is rare to find a field example where the problem is so well-posed, and interpreting outcrop stratigraphy is often limited by non-unique alluvial responses and lack of constraints on key parameters.

Here we revisit a field case in the Piceance Basin, where changes in channel stacking patterns have been attributed to altered hydrological cycling related to the Paleocene-Eocene Thermal Maximum (PETM).
In this location though, the causal relationship connecting hydrological changes to alluvial processes remains unclear, with implications for interpreting the landscape responses to abrupt climate changes.
Using new field observations and paleohydraulic techniques, this study provides additional constraints on alluvial processes in the Piceance Basin before, during, and after the PETM.
We use this new data to offer a refined model for alluvial response to abrupt climate warming, and explore the implications for \pnote[something].
